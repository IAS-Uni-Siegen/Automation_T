\section{Conventional Power Semiconductor Devices}
\title{Conventional Power Semiconductor Devices}  

\begin{frame}[plain]
    \titlepage
\end{frame}

%%%%%%%%%%%%%%%%%%%%%%%%%%%%%%%%%%%%%%%%%%%%%%%%%%%%%%%%%%%%%
%% Homopolar / unipolar machines %%
%% Planned topics: Power MOSFETs, IGBTs, Thyristors %%
%%%%%%%%%%%%%%%%%%%%%%%%%%%%%%%%%%%%%%%%%%%%%%%%%%%%%%%%%%%%%
\begin{frame}{Introduction to Power MOSFETs}
\begin{itemize}
    \item Power MOSFETs (Metal-Oxide-Semiconductor Field Effect Transistors) are widely used in power electronics since the early 1980s.
    \item They offer high on-state current capability and off-state voltage blocking, making them suitable for high-speed switching applications.
    \item Replacing BJTs in many applications, especially those requiring fast switching, due to their voltage-controlled nature.
    \item Understanding the differences between BJTs and MOSFETs is crucial for effective circuit design.
    \item This section introduces the physical structure, voltage and current limits, and failure mechanisms of MOSFETs.
\end{itemize}

\vspace{0.5cm}
\textbf{Figure Placeholder:} \textit{Power MOSFET application context diagram (optional).}
\end{frame}


\begin{frame}{Basic Structure of Power MOSFET}
\begin{itemize}
    \item The MOSFET is constructed with alternating p-type and n-type layers: \textbf{n$^+$-p-n$^-$-n$^+$}.
    \item The n$^+$ (source and drain), p-body, and n$^-$ (drift region) form the core active structure.
    \item Gate terminal is insulated from the body by silicon dioxide (SiO$_2$) forming the gate oxide.
    \item Applying a positive voltage to the gate induces an n-channel in the p-body, allowing current flow.
    \item The drift region (n$^-$) is lightly doped to handle high blocking voltages.
\end{itemize}

\vspace{0.5cm}
\textbf{Figure Placeholder:} \textit{MOSFET cross-section with labels for source, gate, drain, body, and drift region.}
\end{frame}

\begin{frame}{Perspective View of a Power MOSFET}
\begin{itemize}
    \item Power MOSFETs consist of thousands of small identical cells connected in parallel.
    \item Each cell includes source diffusion, gate conductor, field oxide, and drain contact.
    \item High current handling is achieved by maximizing the number of cells and gate width.
    \item This configuration is often referred to as VDMOS (Vertical Diffused MOSFET).
    \item Parasitic elements like a BJT and integral body diode are naturally formed.
\end{itemize}

\vspace{0.5cm}
\textbf{Figure Placeholder:} \textit{Figure 22-1(b): Perspective structure showing MOSFET cellular layout.}
\end{frame}

\begin{frame}{Channel Formation and Operation}
\begin{itemize}
    \item The p-body prevents current flow until a gate voltage is applied.
    \item Applying V$_{GS} > V_{th}$ induces an n-channel in the p-body.
    \item This connects the source to the drain allowing electron flow (n-channel enhancement).
    \item No minority carrier injection occurs—enabling fast switching.
    \item The MOSFET operates as a unidirectional voltage-controlled switch.
\end{itemize}

\vspace{0.5cm}
\textbf{Figure Placeholder:} \textit{Figure 22-2(a): Accumulation layer in ON-state.}
\end{frame}

\begin{frame}{Gate Metallization and Field Plate Action}
\begin{itemize}
    \item Gate metallization overlaps the n$^-$ drift region to serve two purposes:
    \begin{enumerate}
        \item Enhances conductivity during ON-state by forming accumulation layer.
        \item Acts as a field plate to reduce electric field curvature in OFF-state.
    \end{enumerate}
    \item This helps to minimize on-resistance and improves breakdown voltage.
    \item Design strategies include optimizing gate width-to-length ratio to maximize gain.
\end{itemize}

\vspace{0.5cm}
\textbf{Figure Placeholder:} \textit{Figure 22-2(b): Field plate action during OFF-state.}
\end{frame}


\begin{frame}{Inversion Layers and the Field Effect}
\begin{itemize}
    \item The gate region of the MOSFET is composed of:
    \begin{itemize}
        \item Gate metallization (e.g., Al or polysilicon)
        \item Gate oxide ($SiO_2$)
        \item Silicon substrate beneath the oxide
    \end{itemize}
    \item This gate stack acts as a high-quality MOS capacitor.
    \item Applying a small positive $V_{\mathrm{GS}}$ causes:
    \begin{itemize}
        \item Positive charge on gate metal
        \item Negative charge induced on silicon side
        \item Formation of a depletion region by repelling holes
    \end{itemize}
    \item The structure acts as a capacitor inducing a field across the gate oxide.
\end{itemize}
\vspace{0.5em}
\textbf{Note:} The inversion layer forms only after depletion is sufficient.
\end{frame}


\begin{frame}{Formation of Depletion and Inversion Layers}
\begin{columns}
\column{0.55\textwidth}
\begin{itemize}
    \item As $V_{\mathrm{GS}}$ increases:
    \begin{enumerate}
        \item Depletion region widens (Fig. 22-5a)
        \item Electrons begin to accumulate at oxide–silicon interface (Fig. 22-5b)
        \item Inversion layer forms when electron density exceeds hole density (Fig. 22-5c)
    \end{enumerate}
    \item Inversion layer has \textbf{n-type} conductivity and enables conduction between source and drain.
    \item This defines the MOSFET’s enhancement mode behavior.
\end{itemize}
\column{0.4\textwidth}
%\includegraphics[width=\textwidth]{figures/inversion_layer_diagram.png} % Placeholder for Fig. 22-5
\end{columns}
\end{frame}


\begin{frame}{Threshold Voltage and Oxide Capacitance}
\begin{itemize}
    \item Threshold voltage $V_{\mathrm{GS(th)}}$ is the point at which an inversion layer is just formed.
    \item As $V_{\mathrm{GS}} > V_{\mathrm{GS(th)}}$, inversion layer:
    \begin{itemize}
        \item Becomes thicker
        \item Becomes more conductive
        \item Screens the underlying depletion region
    \end{itemize}
    \item Major factor: oxide capacitance per unit area:
    \[
        C_{\mathrm{ox}} = \frac{\varepsilon_{\mathrm{ox}}}{t_{\mathrm{ox}}}
    \]
    \item Other influencing factors:
    \begin{itemize}
        \item Work function difference (metal vs silicon)
        \item Fixed/trapped charge
        \item Body doping and oxide thickness
    \end{itemize}
\end{itemize}
\end{frame}

\begin{frame}{Gate Control of Drain Current Flow}
\begin{itemize}
    \item With $V_{\mathrm{GS}} > V_{\mathrm{GS(th)}}$ and small $V_{\mathrm{DS}}$:
    \begin{itemize}
        \item MOSFET operates in the ohmic (linear) region.
        \item Inversion layer has nearly uniform thickness along the channel.
        \item Drain current $I_D$ increases linearly with $V_{\mathrm{DS}}$.
    \end{itemize}
    \item Voltage drop along the channel varies:
    \[
        V_{\mathrm{ox}}(x) = V_{\mathrm{GS}} - V_{\mathrm{CS}}(x)
    \]
    \item As $V_{\mathrm{DS}}$ increases, $V_{\mathrm{CS}}(x)$ increases toward the drain end.
    \item This leads to non-uniform channel thickness and begins shaping current characteristics.
\end{itemize}
\end{frame}


\begin{frame}{Channel Pinch-Off and Saturation Region}
\begin{itemize}
    \item As $V_{\mathrm{DS}}$ increases such that:
    \[
        V_{\mathrm{GS}} - V_{\mathrm{DS}} = V_{\mathrm{GS(th)}}
    \]
    the channel is \textbf{pinched off} at the drain end.
    \item No inversion layer exists at the drain end—carrier velocity becomes saturated.
    \item Device enters saturation (active) region:
    \begin{itemize}
        \item $I_D$ becomes independent of $V_{\mathrm{DS}}$
        \item Minimum inversion layer thickness is maintained by high electric field
    \end{itemize}
    \item Channel length modulation and velocity saturation effects begin.
\end{itemize}
\end{frame}


\begin{frame}{Electric Field and Velocity Saturation}
\begin{itemize}
    \item At high electric fields, carrier velocity saturates:
    \[
        v_{\mathrm{drift}} \approx 8 \times 10^6~\text{cm/s at } E \approx 1.5 \times 10^4~\text{V/cm}
    \]
    \item This saturation leads to constant $I_D$ in the saturation region:
    \[
        I_D = K (V_{\mathrm{GS}} - V_{\mathrm{GS(th)}})^2
    \]
    with 
    \[
        K = \mu_n C_{\mathrm{ox}} \frac{W}{2L}
    \]
    \item $\mu_n$ is carrier mobility; $W/L$ is width-to-length ratio of the channel.
    \item Design goal: maximize $W/L$ to minimize on-state losses and improve gain.
\end{itemize}
\end{frame}


\begin{frame}{Deviation from Square-Law Model}
\begin{itemize}
    \item Square-law $I_D$–$V_{\mathrm{GS}}$ relationship:
    \[
        I_D \propto (V_{\mathrm{GS}} - V_{\mathrm{GS(th)}})^2
    \]
    \item Breaks down at high $I_D$ due to:
    \begin{itemize}
        \item Velocity saturation (as per $v_{\mathrm{drift}}$ curve)
        \item Reduction in mobility $\mu_n$ due to:
        \begin{itemize}
            \item Increased electric field in inversion layer
            \item Increased electron density near gate
            \item Carrier–carrier scattering effects
        \end{itemize}
    \end{itemize}
    \item At high currents, the relationship becomes approximately linear.
\end{itemize}
\end{frame}
